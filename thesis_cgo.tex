\documentclass[12pt,Bold,letterpaper,TexShade]{mcgilletdclass}

\usepackage{geometry}

\usepackage{texshade}
\usepackage{textcomp}
\usepackage{xspace}
\usepackage{siunitx}
\usepackage{epigraph}
\usepackage{todonotes}
\usepackage{physics}
\usepackage{graphicx}
\usepackage{subfigure}
\usepackage[labelfont=bf]{caption}
\usepackage{amsmath}





\graphicspath{{figures/}}
%\graphicspath{{figures_lowres/}} 

%hires
%\DeclareGraphicsExtensions{.pdf,.png}
%lores
\DeclareGraphicsExtensions{.png,.pdf}

%%%% VARIABLES %%%%%
\newcommand{\diffusion}{$D_{t}$\xspace}
\newcommand{\gct}{$\gamma$-CT\xspace}
\newcommand{\dcunits}{\cm\squared\per\second}
\newcommand{\figref}[1]{{\bf Fig.~\ref{#1}}}
\newcommand{\vek}[1]{\mathbf{#1}}
\newcommand{\vivo}{{\it in vivo}\xspace}
\newcommand{\silico}{{\it in silico}\xspace}
\newcommand{\vitro}{{\it in vitro}\xspace}
\newcommand{\tub}{$\gamma$-Tubulin\xspace}

%%% SETTINGS %%%
\interfootnotelinepenalty=10000

%\usepackage{tex4ht}
%\usepackage{amsmath}
%%%%%%%%%%%%%%%%%%%%%%%%%%%%%%%%%%%%%%%%%%%%%%%%%%%%%
%% Have you configured your TeX system for proper  %%
%% page alignment? See the McGillETD documentation %%
%% for two methods that can be used to control     %%
%% page alignment. One method is demonstrated      %%
%% below. See documentation and the ufalign.tex    %%
%% file for instructions on how to adjust these    %%
%% parameters.                                     %%
\addtolength{\hoffset}{0pt}                        %%
\addtolength{\voffset}{0pt}                        %%
%%                                                 %%
%%%%%%%%%%%%%%%%%%%%%%%%%%%%%%%%%%%%%%%%%%%%%%%%%%%%%
%%       Define student-specific info
\SetTitle{\huge{Molecular Dynamics of the disordered $\gamma$-tubulin carboxyl terminus}}%
\SetAuthor{Carlos G. Oliver}%
\SetDegreeType{Master of Science}%
\SetDepartment{Department of Biology}%
\SetUniversity{McGill University}%
\SetUniversityAddr{Montreal,Quebec}%
\SetThesisDate{\today}%
\SetRequirements{A thesis submitted to McGill University in partial fulfillment of the requirements of the degree of Master of Science}%
\SetCopyright{\textcopyright  Carlos G. Oliver, 2016}%

\makeindex[keylist]
\makeindex[abbr]


%% Input any special commands below
%\newcommand{\Kron}[1]{\ensuremath{\delta_{K}\left(#1\right)}}
\listfiles%
\begin{document}
\maketitle%

\begin{romanPagenumber}{2}%


\SetDedicationName{\MakeUppercase{Dedication}}%
\SetDedicationText{To my grandparents; Hilda Fernandez and Porfirio Oliver. {\it Gracias por su apoyo, amor, y por ser mi inspiraci\'on en todo. Mi \'unica meta en la vida es acercarme un poco a lo que son ustedes. Los quiero mucho.}}%

\Dedication%

\SetAcknowledgeName{\MakeUppercase{Acknowledgements}}%
\SetAcknowledgeText{I give my sincere thanks to my supervisor Dr. Jackie Vogel for her support and mentorship throughout my Master's, and for sharing with me her passion for the complexity of life. I thank my colleagues in the Vogel lab for their friendship, guidance and useful discussions. I would also like to acknowledge the generous financial support from the Cellular Dynamics of Mollecular Complexes fellowship and Dr. Vogel's grants from the	Canadian Institutes for	Health Research and the National Institutes of Health. I thank the Guillimin support team for their invaluable technical help in working with the Calcul Qu\'ebec compute clusters. I also thank Vladimir Reinharz for his help in with computational problems, and for all the enriching discussions.
	\newline \indent  I would like to express my deepest gratitude to all those whom I am so fortunate to call my friends and loved ones. To my brother, whose friendship means everything. To my father for his care and support. And to my mother; for being there for me every day, in every way despite it all, for being my role model and my best friend.}%
\Acknowledge%

\SetContributionName{\MakeUppercase{Contribution of Authors}}%
\SetContributionText{All data collection and results obtained during my M.Sc. are presented here as a standard format thesis. This work includes results and figures obtained by collaborators in the Department of Chemistry Dr. Anthony Mittermaier and Jason Harris who conducted NMR experiments and generated all NMR based figures. All computer simulation work and MD figures were done by myself. Collection of tubulin primary sequences was done by Roman Sarrazin Gendron. All text and literature review in this thesis was done by myself with feedback from my supervisor Jackie Vogel and collaborator Anthony Mittermaier. }%
\Contribution%


%%%%%%%%%%%%%%%%%%%%%%%%%%%%%%%%%%%%%%%%%%%%%%%%%%%%%
%%         English Abstract                        %%
%%%%%%%%%%%%%%%%%%%%%%%%%%%%%%%%%%%%%%%%%%%%%%%%%%%%%
\SetAbstractEnName{\MakeUppercase{Abstract}}%
\SetAbstractEnText{With recent advances in experimental and computational methods in structural biology, it is becoming increasingly clear that protein function is not only dependent on stable architectures, but equally so on the absence of well-ordered domains. These elements, also known as intrinsically disordered proteins (IDPs) or regions (IDRs) are protein chains that do not adopt organized three dimensional structures, but have nonetheless been shown to be highly functional. Instead, due to their flexible backbones, IDPs/IDRs are able to explore a large ensemble of conformations, resulting in an expansion of the functional repertoire of proteins. The cell is able to harness this flexibility by targeting IDRs for post-translational modifications (PTMs) such as phosphorylation. By altering the conformational sampling of IDRs through PTMs, disordered elements serve as a key points for functional control and signal integration. This paradigm is crucial for ensuring the proper execution of complex cellular processes that involve a large number of multi-tasking proteins that need to act in a coordinated manner. However, the physical mechanisms linking IDRs to functional output remain largely unknown. In this work, we study the role of phosphorylation in the assembly of the mitotic spindle which effects chromosome segregation during cell division. Phosphorylation at Tyrosine 11 (Y11) in the intrinsically disordered C-terminus of \tub (\gct) has been shown to control key aspects of the assembly of microtubules in the mitotic spindle and leads to a temperature sensitive growth phenotype in budding yeast. Through Molecular Dynamics computer simulations (MD) and Nuclear Magnetic Resonance spectroscopy (NMR) we show that non-phosphomimicking and phospho-mimicking states of the \gct sample disordered and collapsed conformations. However, the phosphomimicking (Tyrosine 11 to Aspartic Acid) mutant undergoes switch-like collective motions to an extended but also disordered state. We propose that this transition serves to control the binding of proteins involved in shaping microtubule dynamics. This is the first observation of switch-like behaviour in IDRs/IDPs where both states are disordered, making this a novel physical mechanism of control in with the potential to regulate cellular function.}

\AbstractEn%

%%%%%%%%%%%%%%%%%%%%%%%%%%%%%%%%%%%%%%%%%%%%%%%%%%%%%
%%         French Abstract                         %%
%%%%%%%%%%%%%%%%%%%%%%%%%%%%%%%%%%%%%%%%%%%%%%%%%%%%%
\SetAbstractFrName{\MakeUppercase{ABR\'{E}G\'{E}}}%
\SetAbstractFrText{ The text of the abstract in French begins here.  }%
\AbstractFr%

\TOCHeading{\MakeUppercase{Table of Contents}}%
\LOFHeading{\MakeUppercase{List of Figures}}%
\tableofcontents %
\listoffigures %

\index[abbr]{MD@MD: Molecular Dynamics}
\index[abbr]{\gct@\gct: $\gamma$-Tubulin C-Terminus}
\index[abbr]{NMR@NMR: Nuclear Magnetic Resonance}
\index[abbr]{$\gamma$-TuRC@$\gamma$-TuRC: $\gamma$-Tubulin Ring Complex} 
\index[abbr]{\diffusion@\diffusion: Translational Diffusion Coefficient}
\index[abbr]{WT@WT: Wild-Type}
\index[abbr]{YD@YD: Y445D mutant}
\index[abbr]{RMSD@RMSD: Root Mean Square Displacement}
\index[abbr]{$R_g$@$R_g$: Radius of Gyration}
\index[abbr]{ns@ns: Nanosecond}
\index[abbr]{$\mu$s@$\mu$s: Microsecond}
\index[abbr]{IDP@IDP: Intrinsically Disordered Protein}
\index[abbr]{IDR@IDR: Intrinsically Disordered Region}
\index[abbr]{CREB@CREB: Cyclic AMP Response Element Binding Protein}
\index[abbr]{CBP@CBP: CREB Binding Protein}
\index[abbr]{KID@KID: Kinase Inducible Domain}
\index[abbr]{DNA@DNA: Deoxyribonucleic Acid}
\index[abbr]{GRIPS@GRIPS: $\gamma$-Tubulin Ring Proteins}
\index[abbr]{PTM@PTM: Post-Translational Modification}
\index[abbr]{NVE@NVE: Microcanonical Ensemble}
\index[abbr]{NVT@NVT: Canonical Ensemble}
\index[abbr]{NPT@NPT: Isothermal-isobaric Ensemble}
\index[abbr]{OPLS-AA@OPLS-AA: Optimized Potential for Liquid Simulations - All Atom}
\index[abbr]{SPCE@SPCE: Extended Single Point Charge}
\index[abbr]{GROMACS@GROMACS: GROnigen Machine for Chemical Simulatoins}
\index[abbr]{VMD@VMD: Visual Molecular Dynamics}


\printindex[abbr]{KEY TO ABBREVIATIONS}{KEY TO ABBREVIATIONS}{}

\end{romanPagenumber}

%\mainmatter %

%introduction chapter
%!TEX root = thesis_cgo.tex
\chapter{Introduction}


\epigraph{Everything existing in the universe is the fruit of chance and necessity.}{Democritus}

Molecular  machines  are  assemblies of proteins and associated molecules that  work together  in a coordinated manner  to solve a biological problem.  These biological problems, such as coordinating chromosome segregation  during cell division, regulation of cell cycle timing, execution of protein  synthesis, etc.  encompass many processes essential to life. Given the necessity for survival in the face of ever changing environments,  evolution has produced a large diversity of molecular mechanisms for solving these problems in a flexible yet robust  manner.  A biological machine that  only functions properly under a narrow range of  conditions is less likely to support the life of a single individual or a population.  Therefore,  at the core of each of those processes are highly complex networks of proteins that  are able to assemble, communicate, coordinate,  self-regulate and self-correct in order to accomplish the necessary task reliably.  For example, the vital process of DNA replication is executed by a large multitude of proteins that  each contribute to the process of copying the genome ~\cite{bell2002dna}. The replication  machinery must first read environmental  cues for initiating  replication  at the correct time. Meanwhile, complex combinatoric  signaling networks ensure that  DNA replication  unfolds in a processive manner,  enzymatic components  perform physical work to unwind DNA strand  for copying, and others  communicate  with the DNA repair machinery  to correct copying errors and avoid harmful mutations. It is clear that  solving the biological problem of DNA replication requires the ability of participating proteins to interact  with many different partners  and mediate  many different processes.


The structure and function of each protein is encoded in its unique sequence, or chain, of amino acids. Physical interactions between amino acids give rise to a specific 3D arrangement of the protein chain, also known as structure. The structure of each protein allows for specific interactions between  proteins  to  assemble molecular machines, recruit  necessary factors and mediate chemical reactions. See \figref{fig:tub4} for a visualization of protein structure. Since the 1950s when the first X-ray crystallography protein structure was solved ~\cite{kendrew1958three}, we have learned a great deal about how 3D architecture and conformational sampling of the chains give rise to protein function. X-ray crystallography accesses atomic-scale conformations of folded protein  domains, allowing us to infer that  coordinated  motions between structural conformations is the main element of control in protein function ~\cite{hegyi1999relationship}. For example, X-ray crystallography experiments have shown that the activity of Calmodulin, an important signalling protein, is modulated by conformational changes of its $\alpha$-helix domains brought about by binding of Ca\textsuperscript{2+} ions~\cite{meador1992target}. These conformational rearrangements initiate a clamping motion of the helical domains which allow Calmodulin to bind to its downstream targets. However, it is important to note that X-ray crystallography only offers static pictures of protein structure, and provides information mostly on the spatial arrangement of relatively large and stable domains. It therefore became a long standing dogma that the stable 3D folds of a protein chain dictate a protein's function, also known as the "one structure - one function" paradigm. 

However, as we saw with DNA replication, a single protein is often required to fulfill many functions, and interact with various different partners. It is therefore unlikely that such large scale and consequently slow structural motions can account for all of the precise and rapid control we observe in biological systems. A static description of proteins is not sufficient to explain the degree of functional flexibility and control that we observe. This leaves us with several questions. If one structure means one function, how can the same protein fulfill multiple functions and engage in many different interactions? How can molecular machines offer such precise control of functionality while counting only on a static architectures? The broad aim of this thesis is therefore to improve our understanding of the physical mechanisms underlying the functional complexity of molecular machines.

\begin{figure}[h!]
\centering
\includegraphics[height=0.4\textheight]{tub4}
\FigureCaption{$\gamma$-Tubulin 3D Structure}{Visual representation of the yeast $\gamma$-Tubulin protein based on the human $\gamma$-Tubulin structure derived from X-ray crystallography~\cite{aldaz2005insights}. The protein is composed of a highly ordered globular domain stabilized by various $\alpha$-helices and $\beta$-sheet domains and measures \SI{2.20}{\nm} in radius of gyration. In red, we see the disordered \gct region which was absent in the crystal structure and was modeled by \texttt{RaptorX} ~\cite{kallberg2012template}.}
\label{fig:tub4}
\end{figure}

\section{Disorder in proteins}

In recent years, it has been recognized that functional plasticity  can be found in regions of proteins that  do not adopt stable architectures, also known as intrinsically disordered regions (IDRs). While IDRs are highly flexible and largely unstructured, functional studies have shown that they are necessary for many cellular processes ~\cite{wright2015intrinsically}.  \footnote{Some works make the distinction between IDP and IDR where an IDR is an intrinsically disordered region and IDP is a fully disordered protein.} Instead of relying on a structure for function, IDR functionality lies the absence of structure. This flexibility offers the protein rapid access to a vast pool of conformations with which to fine-tune and diversify its function.  

The study of IDRs is relatively new to structural biology. This is largely due to the fact that the main tool being used for structural biology in the past decades, X-ray crystallography, fails to detect patterns in unfolded chains, making it difficult to study highly dynamic elements in protein such as IDRs. Unstable protein domains such as IDRs that sample many conformations produce averaged out electron scatter patterns that cannot be interpreted ~\cite{putnam2007x}. Techniques that  do produce information  on dynamics,  such as Nuclear Magnetic Resonance (NMR) only developed for proteins  until 1984 ~\cite{wuthrich2001way}, 25 years after the first structure was solved by X-ray crystallography  in 1958 ~\cite{kendrew1958three}.  Another approach for studying protein dynamics with atomic resolution is through computational simulation.  Physical models of proteins whose motions are computed \silico have been shown to provide important information regarding detailed dynamics of biomolecules ~\cite{karplus2002molecular}. However, until recently, these techniques,such as Molecular Dynamics (MD), were greatly limited by shortcomings in computer power. However, with large advances  in experimental  and computational techniques  in recent years, we have been able to study the dynamic properties  of IDPs in great detail, and have found that  they play key roles in the control of molecular machines.

Over 15,000 proteins in the Protein Data Bank have been predicted to contain long disordered regions ~\cite{romero1998thousands}; it is therefore not surprising that  IDPs have also been implicated  in a multitude of cellular processes and disease states ~\cite{uversky2008intrinsically}.  Interestingly, it has been shown that viral proteins use IDPs in their proteins to hijack cellular proteins and use the flexibility of IDPs to mimic host proteins and recruit host cellular machinery in order to propagate.\cite{davey2011viruses} This would suggest that viral proteins use IDPs to make efficient use of their smaller genomes and obtain a greater range of function from the limited number of proteins in their genomes. It is now clear that IDPs, through their lack of structure, are an important adaptive feature that drive the functional complexity and robustness we observe in molecular machines. \\

\section{Physical mechanisms of IDR function in cellular machines}

In this section we will give a brief account of some of the physical mechanisms of IDR function that have been described in the literature. \\

{\it Phosphorylation}

A key aspect of dynamic control is the ability to modulate function in a precise and reversible manner.  The cell needs to be able to induce and inhibit interactions in a time and space dependent manner. To solve this problem, the cell harnesses the structural malleability of IDRs/IDPs by coupling these elements with post translational modifications, most commonly, phosphorylation. Phosphorylation is the reversible covalent addition of a phosphate  group effected by a protein kinase,  which carries a negative  charge to a tyrosine, serine or threonine  amino acid. The reverse reaction is catalyzed by enzymes called phosphatases which remove the phosphate group. The addition of a phosphate group introduces the potential inter and intra-molecular hydrogen bonding which alters the electrostatic environment of the IDP/IDR ~\cite{van1990effect}. This change can in turn bias the stochastic conformational sampling of the IDP/IDR in a particular direction. Because phosphorylation is reversible, it acts as a means for driving structural switching which can then be used to modulate a large range of interactions ~\cite{kern1999structure, ramelot2001phosphorylation, oxley2008integrin}. Not surprisingly, it has been seen in many studies that IDPs are prime targets for phosphorylation ~\cite{iakoucheva2004importance}.  The use of phosphorylation as an information carrier has been described in various cellular systems. For example, in cell cycle control, there is a strong need for a specific temporal sequence of interactions to be enforced ~\cite{yoon2012cell}. In such a case, the sequential phosphorylation of a single target modifies the affinity for the same target to the next target in the pathway, ensuring that interactions take place in an ordered manner ~\cite{wright2015intrinsically}. Phosphorylation can also be used to enforce thresholds, where in order to avoid the negative impact of accidental interactions, certain interactions will be blocked until an IDP/IDR has achieved a certain number of phosphorylations ~\cite{nash2001multisite}. Having accumulated enough phosphorylations, a structural rearrangement favours the interaction. 

{\it Disorder-Order transitions}

\todo{clarify entropy-enthalpy}

The best explored physical mechanism of IDP function is the fold-on binding paradigm ~\cite{wright2009linking}. In this case, IDPs in the free form are unstructured, and when they encounter their binding target, they undergo a folding transition (disorder to order) to form a stable complex. The lack of structure in the unbound state allows the the IDP/IDR the necessary flexibility to recognize multiple targets, and it allows binding to be inducible instead of constitutive. A well studied example of this kind of mechanism is the binding of the transcriptional activator protein CREB and its co-activator CBP ~\cite{gianni2012folding}. An IDR in CREB known as KID mediates binding to CBP where upon binding, the IDP folds into a pair of helices. However, this binding process is not favoured spontaneously due to a high entropic barrier. \footnote{Protein folding is an ordering of the protein backbone which typically implies a loss of entropy to the system and is therefore energetically disfavoursed, this is the entropic barrier; $\Delta G = \Delta H - T\Delta S$, where $\Delta H$ is the change in enthalpy, or heat energy available to the system, and $\Delta S$ is the entropy, or degree of disorder/conformational freedom available to the system. However, if the folding process releases sufficient heat ($\Delta H$), the loss in entropy is overcome and the folding reaction is energetically favoured; $\Delta G < 0$. The release in heat by the folding of the protein typically increases the entropy of the surrounding water molecules thus preserving the second law of thermodynamics.}However, when the KID is phosphorylated, the phosphoryl group interacts with CBP by forming hydrogen bonds which result in a negative enthalpic change that compensates for the loss of entropy and thus makes the folding reaction favourable. Because of the inducible nature of this interaction, CBP is able to also interact with other co-factors, which has been reported in the literature \cite{radhakrishnan1997solution}. This is an example of how even though the association state of the IDP is ordered, it is  the intrinsic disorder and entropy of the unbound state which is the key for controlling the interaction. \\


\pagebreak

{\it \lq Fuzzy\rq interactions}

\par Disorder to order transitions are not necessary for IDPs to confer functionality. There are a growing number of examples where IDPs/IDRs are involved in functional interactions while remaining in a disordered state \cite{tompa2008fuzzy}. Such interactions have been labeled with the term \lq fuzzy\rq as they maintain a heterogenous conformational ensemble throughout their lifetimes. There exist various physical mechanisms by which fuzziness, or disorder, in binding interactions confers advantages to protein function. For example, binding interactions between an IDP/IDR and a target protein where the IDP/IDR is able to form alternate contacts with its binding target can help reduce the entropic cost of binding, as well as control the accessibility of different sites on the protein for modulating interactions with different targets \cite{graham2001tcf4, fontes2000structural}. IDPs/IDRs can also play a role in interactions without making direct contacts with the binding partner by acting as flexible linkers for folded domains \cite{bhattacharyya2006ste5}, or as \lq antennae\rq for \cite{sigalov2004homooligomerization} recruiting further interactiors and stabilizing the binding of folded domains through long range interactions \cite{zor2002roles, yu1994structural} 

It is becoming increasingly evident that nature has harnessed disorder as an adaptive mechanism for control in protein-function. High flexibility in protein conformational state allows switch-like control over interactions and activity, fine-tuning the kinetics of interactions, precise signal integration, controlled multiple partner binding, etc.  It is due to these biophysical properties that the cell is able to carry out its complex tasks with such robustness and precision. Advances in this field have caused us to reconsider the  \lq one structure, one function\rq paradigm that has prevailed in structural biology for decades. However, this remains a relatively novel area of structural biology, and there still remain many unsolved physical mechanisms in IDPs/IDRs.

\section{IDP function in the mitotic spindle}
 
In this section we will address the role of IDRs in controlling the function of the mitotic spindle. The mitotic spindle is a complex molecular machine composed of microtubules, force generators, chromosomes, and numerous effector molecules which act in a coordinated manner to accomplish the process of chromosome segregation during cell division ~\cite{karsenti2001mitotic}. This process ensures that genetic material is transferred from the mother to the daughter cell in a timely manner, and without errors which would in most cases result in fatality. Because cell division is a fundamental task in every cell's lifetime, mitotic spindles share common design features throughout eukaryotes ~\cite{kubai1976evolution}. The  task  of properly  arranging  and  segregating  chromosomes  is effected ultimately  by hollow \SI{25}{\nm} filaments composed of polymerized tubulin, known as microtubules,  which attach to chromosomes and exert forces to move the  DNA into mother and daughter  cells ~\cite{kline2004mitotic}.  In order to study the underlying mechanisms at play, we work with the mitotic spindle of the budding yeast {\it Saccharomyces cerevisiae} due to its minimal, yet highly conserved construction ~\cite{kubai1976evolution}. 

\subsection{Microtubules}

Microtubules are constructed of alternating pairs, or heterdimers, of the globular proteins $\alpha$ and $\beta$ tubulin. A cylindrical  arrangement  of tubulin  dimers results  in the  formation  of a microtubule, whose rigidity can be adapted in a length dependent manner.   By adding or losing subunits, microtubules can increase and decrease in length, and push or pull directly on targets.  Microtubules also bind proteins that link them to chromosomes, to vesicles, and even to each other when forming microtubule bundles.  ~\cite{dogterom2005force}. Microtubules  have a number  of other  functions,  such  as serving as roadways  along which transport molecules can carry cargo, and form an adaptable “cytoskeleton” that contributes to cell shape and movement. The spontaneous assembly of tubulin dimers in solution into a microtubule is heavily disfavoured. However, when free floating tubulins encounter pre-formed “nucleus” or microtubule seed, the growth of a microtubule is greatly facilitated  ~\cite{joshi1992gamma}. In cells, this template is known as the $\gamma$-Tubulin Ring Complex ($\gamma$-TuRC) which is a ring-like assembly of $\gamma$-Tubulin molecules held together by various other proteins known as $\gamma$-tubulin ring proteins (GRIPs) ~\cite{kollman2011microtubule}. $\gamma$ tubulin shares a similar structure to $\alpha$ and $\beta$ tubulin and have been shown to act as nucleation templates for microtubules when assembled in a ring complex ~\cite{kollman2011microtubule}. 

\subsection{$\gamma$-Tubulin \& the $\gamma$-CT}

Microtubule nucleation was thought to be the sole function of $\gamma$ tubulin for many years. However, recent evidence from budding yeast suggests that $\gamma$-tubulin might have additional functions in controlling microtubule properties ~\cite{vogel2000carboxy,vogel2001phosphorylation, cuschieri2006gamma, nazarova2013cdk1}. $\gamma$-tubulin bound to the spindle poles is phosphorylated in vivo at  8 sites ~\cite{vogel2001phosphorylation,keck2011cell}.
Several functional studies following up on the finding that $\gamma$-tubulin is regulated show that mutations altering phosphorylation sites, all of which lie in IDRs, have consequences on the organization and stability of microtubules but no effect on their nucleation. This opens a relatively unexplored field of functional coupling whereby the $\gamma$-TuRC acts not only as a microtubule nucleator, but also as a signal integration hub for regulating downstream events in microtubule organization.

One phosphorylation site in \tub that has a important role in spindle function is the highly conserved  tyrosine  (Y)  445  which lies in the  disordered  carboxyl  terminal  tail  of \tub  (\gct). The \gct{} is defined as the final 35 residues in the C-terminal portion of $\gamma$-tubulin, which lies outside of the folded globular domain. The  \gct is essential for survival in budding yeast  ~\cite{vogel2000carboxy}. Substitution of an aspartic or glutamic acid (D/E) residue in the place of Y445 results  in slow growing cells with unstable  and misaligned mitotic  spindles ~\cite{vogel2001phosphorylation}. The Y445D/E substitutions are used as a means for constiuatively mimicking the electrostatic  environment  of a phosphate  group  by introducing  a negative  charge. Defects in spindle function observed in these mutants suggest that phosphorylation is limited to a specific stage of the cell cycle and/or subset of molecules. However, very little  is known about  the interactions and physical mechanisms that  phosphorylation  of  Y445 may control.   The  coupling of post-translational modifications  (PTMs) to the C-terminal  tails of tubulins  is well described in the $\gamma$-tubulin orthologues  $\alpha$ and $\beta$ tubulin.  Specific combinations  of PTMs  on the $\alpha$ and $\beta$ tubulin  tails act as a  \lq tubulin code\rq for selectively recruiting  motor  proteins  and  microtubule  associated  proteins to the microtubule  lattice.  A similar code has not yet been described for $\gamma$-tubulin despite evidence that  is is regulated \vivo.  While the functional importance  of phosphorylation and IDRs in $\gamma$-tubulin is becoming increasingly clear, the physical mechanisms  by which local modifications  at  IDRs can have global impacts  on the large molecular machine remain unstudied.


 \clearpage 
 
 \section{Experimental question}
 
We hypothesize that phosphorylation of the \gct IDR is a key event in the regulation of microtubule dynamics which allows cells precise control over the building of the mitotic spindle. Moreover we propose that such control is achieved via the local  addition of negative charge modulates the global dynamics and conformational sampling of the \gct.
 
 
 \section{Approach}
 

In order to study changes in conformational sampling of the \gct, we use a powerful computational technique known as Molecular Dynamics (MD) simulations. We will be simulating the dynamics of two forms of the \gct: WT, and Y11D (Y445 in the full protein). We will perform simulations on the \gct in isolation as well as in the presence of the entire $\gamma$-Tubulin protein. Analysis of MD simulations will be guided and validated by NMR experiments on the same system performed by collaborators.

 
 
 
 
  

%methods chapter
%!TEX root = thesis_cgo.tex
\chapter{Theory \& Methods}
\epigraph{Life can only be understood backwards; but it must be lived forwards.}{S{\o}ren Kierkegaard}

The main technique I will use to study the behaviour of IDPs/IDRs is the computational technique of Molecular Dynamics (MD) simulations ~\cite{haile1992molecular}. Protein dynamics are shaped by various types of physical interactions, acting on many conformational degrees of freedom, and on timescales spanning femtoseconds to milliseconds. This level of complexity makes it very challenging to predict the dynamics, or compute ensemble quantities of IDPs {\it ab initio} . MD is a brute-force approach which addresses this problem by iteratively solving the equations of motion for every interaction in a system of atoms in 3D space.  What results is a trajectory and a velocity for every atom in the system in time, which we can use to visualize the conformational sampling of our IDP of interest and compute thermodynamic quantities. While this can be a computationally demanding task, it is currently the most reliable way of studying protein conformational sampling \silico and has been succesfully applied to many biomolecular systems ~\cite{karplus2002molecular}. We will use this approach to study the conformational dynamics of two isoforms of the \gct: WT, and Y11D.

\section{Molecular Dynamics Simulations}

We represent our system as a set of $N$ atoms represented as  vectors $R = \{\vek{r_1}, \vek{r_2}, ... , \vek{r_N}\}$ in three dimensional space. We then use classical Newtonian mechanics to obtain the changes in position of the atoms as a function of time. For a peptide in solution, this system would consist of the atoms in the peptide, ions, water atoms, and the forces arising from interactions between them. 

\subsection{Computing atomic trajectories}

MD is centered on the principle that the potential energy arising from interacting particles is a function of their positions in space. Given a function describing the potentials arising from interactions between the different atoms, which we call a force field, we can iterate through every atom in the system and calculate resulting forces as a function of potential energy. The potential energy given by the force field can be written as $V(\vek{r_1}, \vek{r_2}, ..., \vek{r_N})$ and is a function of the positions of each atom. Using the classical definition of force as $\vek{F} = m\vek{a}$, we can combine the positions of each atom with the force field to compute the force acting on each atom as follows.  

\begin{equation}
\vek{F_i}  = -\frac{\partial V(\vek{r_1}, \vek{r_2}, ..., \vek{r_i}, ... \vek{r_N)}}{\partial \vek{r_i}} 
 \end{equation}

 Given that the force on an atom is the result of interactions with all other atoms in the system, we obtain the force on a particular atom as the sum of the force of the interactions with all other atoms $j$ in the system,  $\vek{F_i} = \sum_{j} \vek{F_{ij}}$ Given the total force on an atom, we can compute its trajectory in space by numerically integrating Newton's equations of motion. This process is repeated and trajectories are stored and updated for the desired number of steps in the simulation.
 
 \begin{equation}
 \frac{\partial^2 \vek{r_{i}}}{\partial t^2} = \frac{\vek{F_i}}{m_i}
\end{equation}

\subsection{Force Field}

The functions for potential energy of every type of interaction in the system are defined in what we call a force field. The energy between two interacting atoms can be broken down into two broad types of interactions: bonded and non-bonded interactions.

\begin{equation}
E_{total} = E_{bonded} + E_{non bonded}
\end{equation}

The bonded energy term can be written as the sum of energies arising from the bond itself  ($E_{bond}$) which is a function of the bond length, the potential arising from the angle formed by the bond ($E_{angle}$), as well as the torsional/dihedral angle ($E_{dihedral}$) arising from the rotation of three bonds about two intersecting planes.

Non-bonded interactions can have two contributing factors; electrostatic force, and van der Waals force. The electrostatic potential ($E_{electrostatic}$) arises from the interaction of the charges of particles, while the van der Waals potential  ($E_{van der Waals}$ arises from the attraction or repulsion between uncharged groups. Combining all of these terms, we can write the full description of forces in the system as:

\begin{equation}
E(\vek{r_i},, ... \vek{r_{N}}) = E_{bond} + E_{angle} + E_{dihedral} + E_{van der Waals} + E_{electrostatic} 
\end{equation}


The MD algorithm evaluates $E(r_{N})$ at every time step to obtain the force on each atom, and therefore the trajectory at each time step. Given this low-level description of the system, more complex phenomena such as the hydrophobic effect and hydrogen bonding which are known to be essential to protein dynamics do not need to be coded explicitly in the models. Instead, they arise naturally from the definition of the system.

Another key component to the force field is the definition of parameters for the different types of interactions and particles in the system. Key parameters include values for charge, mass, bond length, etc. and are obtained from experimental measurements. The force field must naturally also contain a set of definitions for the various types of atoms and functional groups it can model. Therefore, the choice of force field can have important consequences on the outcome of the simulations and must be chosen with care.

In this work, we will be using the Optimized Potentials for Liquid Simulations - All Atom (OPLS-AA) force field ~\cite{jorgensen1988opls} that can be represented as:


\begin{equation}
E_{bond} = \sum_{bonds} K_{r} (\vek{r} - \vek{r_{0}})^2	
\end{equation}

\begin{equation}
 E_{angle} = \sum_{angles} k_{\theta} (\theta - \theta_0) 
\end{equation}

\begin{multline}
E_{dihedrals} = \sum_\mathrm{dihedrals} \Big( \frac {V_1} {2} \big[ 1 + \cos (\phi-\phi_1) \big] 
                + \frac {V_2} {2} \big[ 1 - \cos (2\phi-\phi_2) \big] \\
                + \frac {V_3} {2} \big[ 1 + \cos (3\phi-\phi_3) \big] 
                + \frac {V_4} {2} \big[ 1 - \cos (4\phi-\phi_4) \big] \Big)
\end{multline}

Where $\phi$ is the dihedral angle and $V_{i}$ are the coeffiecients in the Fourier series. Non-bonded energies are computed as follows:

\begin{equation}
E_{nonbonded} = \sum_{i>j} f_{ij} \left(
                    \frac {A_{ij}}{r_{ij}^{12}} - \frac {C_{ij}}{r_{ij}^6}
                    + \frac {q_iq_j e^2}{4\pi\epsilon_0 r_{ij}} \right)
\end{equation}

Where $A$ and $C$ represent combining rules which allow us to obtain the interaction energy of dissimilar non-bonded atoms. OPLS uses standard combining rules where $A_ij = \sqrt{A_{ii}A_{jj}}$ and $C_{ij} = \sqrt{C_{ii}C_{jj}}$ ~\cite{good1970new}.

We believe the OPLS-AA force field to be the best fit for modeling charge-dependent motions of the \gct. Whereas other popular force fields such as CHARMM were parametrized on X-ray crystallographic experimental values of folded globular proteins, the OPLS force field was optimized with quantum chemical calculations of charged short peptides ~\cite{kukol2008molecular}. This approach is likely to more accurately model the dynamics of unfolded peptides where other force fields may tend to favour the formation of collapsed stable secondary structure motifs ~\cite{henriques2015molecular, tran2008role}. Furthermore, it has been reported that the small and localized treatment of charged groups in OPLS-AA is well suited for systems where local charge interactions drive global dynamics ~\cite{vitalis2009absinth}. We show that a simulation under OPLS-AA is able to accurately produce an extended conformation for the \lq molecular ruler \rq polyproline ~\cite{schuler2005polyproline} \figref{fig:pp}. We also control the appropriateness of the force field by comparing MD values to NMR measurements on the same system.

\begin{figure}
\centering
\includegraphics[width=0.6\textwidth]{pp}
\FigureCaption{Polyproline Simulation}{Simulation of 39-mer polyproline peptide known to be fully extended ~\cite{schuler2005polyproline} behaves as expected and accesses high radius of gyration conformations in a $1 \mu s$ simulation using the OPLS-AA force field.}
\label{fig:pp}
\end{figure}


\subsection{Preparing the system for a simulation}

The starting point of an MD simulation is a force field and a set of initial atomic coordinates for the system of interest. Before a simulation can be successfully run, there are several pre-processing steps that must be executed. 

Hydrogen bonding and the hydrophobic effect play very important roles in shaping the dynamics of polypeptides therefore, the model must include water molecules. We place the peptide atoms in a simulated box under periodic boundary conditions where water molecules are introduced to fill the remaining space. All subsequent force calculations in MD will consider solvent-solvent and solvent-peptide atomic interactions. We use an explicit water regime, meaning that all solvent atoms are modeled as discrete units in the system. While faster alternatives to this paradigm which represent the solvent with mean field behaviour, known as implicit solvent models are available, it is well documented that an explicit treatment currently provides results with the highest accuracy ~\cite{onufriev2008implicit, arnold1994evaluation, zhou2003free}. 

Once the peptide is solvated, any initial steric clashes between atoms must be allowed to relax. Typically this involves executing an energy minimization algorithm which searches for atomic coordinates that minimize the forces between atoms to move the system towards an energy minimum. No minimization algorithm guarantees convergence to a global minimum in finite time on a realistic system. However, convergence to a local minimum is often sufficient to eliminate significant clashes. In this work, we use the steepest descent energy minimization algorithm implemented in GROMACS ~\cite{hess2008gromacs}. 

At this point, we could begin an MD simulation and obtain trajectories in the NVE ensemble (constant number of particles, volume, and energy). However, we are often interested in comparing results from MD to experimental measurements such as those from NMR where the system is under constant temperature and pressure. It is therefore necessary to ensure that the forces in the system do not produce large fluctuations in the pressure and temperature of the ensemble. In order to keep the temperature constant and achieve an NVT sampling (constant number of particles, volume, and temperature) we use a thermostat. Since the temperature of a system is a function of the kinetic energy, a thermostat re-scales the velocities of the atoms in the system to achieve a given temperature. Likewise, for maintaining constant pressure, a barostat adjusts the size of the simulation box to counteract fluctuations in pressure and thus achieving an NPT ensemble (constant number of particles, pressure, and temperature). During the equilibration step, we first let the system adjust to the desired temperature by executing a short simulation in NVT. Then under NPT we allow the system to adjust to the desired pressure. Once both equilibration simulations are complete, the system is ready for a full simulation.

\section{Trajectory Analysis}

The MD simulation generates a set of coordinates for every atom in the system as a function of time, $\vek{r(t)}$. From these trajectories we can compute several quantities to study conformational changes in the peptide over time. 

\subsection{Root Mean Square Deviation}

We measure the square displacement between the coordinates of atom $i$ at time $t$ weighted by the mass of the atom $m_i$. We iterate this process for every atom in the peptide to obtain a measure of the degree of change between two conformations in time weighted by atomic masses in matrix $M$.

\begin{equation}
\text{RMSD}(t_1, t_2) = \bigg[ M^{-1} \sum_{i=1}^{N} m_{i} \lvert \lvert \vek{r_{i}}(t_{1}) - \vek{r_{i}}(t_{2}) \rvert \rvert^2 \bigg]^\frac{1}{2}
\end{equation}

\subsection{Radius of gyration}

The radius of gyration is a measure of a structure's compactness. To obtain the radius of gyration, we compute the mean squared distance from the position vector $\vek{r_i}$ to the molecule's centre of mass $r_{mean}$.

\begin{equation}
R_g(\vek{r}) = \sqrt{N^{-1} \sum_{k=1}^{N} (\vek{r_k} - \vek{r_{mean}})^2}
\end{equation}

\subsection{Diffusion coefficient and Hydrodynamic Radius}

Like radius of gyration, diffusion coefficient (\diffusion) and hydrodynamic radius ($R_h$) is a proxy for the compactness of a macromolecule. However, \diffusion and $R_h$ are quantities that describe the size of the molecule in the context of their solvent. Because biomolecules perform all of their function in solution, and are shaped by their interactions with the solvent, these quantities are often more informative than $R_g$ for our purposes.

The translational diffusion coefficeint of a macromolecule is defined as the rate at which its center of mass is able to diffuse through a solvent of a given viscosity under a certain hydrodynamic model. Conformations with high diffusion rates experience rapid displacement of their center of mass, while conformations with greater viscous force with the solvent experience reduced diffusion coefficients. 

The process of computing the translational diffusion coefficient for a particular conformation is done by the software package \texttt{hydroNMR} ~\cite{de2000hydronmr}. The method of calculating the diffusion coefficient will not be discussed here in detail as it is beyond the scope of this work. The main concept is that the software models each atom in the system, in our case a trajectory obtained by MD, as a spherical bead. The resulting chain of beads is packed into a hexagonal lattice and internal beads are removed to extract a topology of residues exposed to the solvent. From this topology, the software calculates the frictional force that a given conformation would exert on the solvent, this quantity is contained in the translational friction tensor $\Xi$. The following expression gives us the translational diffusion tensor $\vek{D_t}$.

\begin{equation}
\vek{D_{t}} = k_BT\Xi_{t}^{-1}
\end{equation}

Where $k_B$ is the Boltzmann constant (\SI{1.380e-23}{\square\metre\kilo\gram\per\square\second\per\kelvin}) and T is the temperature in Kelvin. The trace of the translational diffusoin tensor is invariant regardless of the molecule's orientation and is thus used to define the translational diffusion coefficient $D_t$.

\begin{equation}
D_t = \frac{1}{3}tr(\vek{D_t})
\end{equation}

Knowing \diffusion, we can use the Stokes-Einstein equation to obtain an expression for the effective hydrodynamic radius which measures the diffusion of spherical particles in solvent.

\begin{equation}
R_h = \frac{k_{B}T}{6\pi \eta D_{t}}
\label{eq:stokes}
\end{equation}

Where $R_h$ is the hydrodynamic radius and $\eta$ is the viscosity of the solvent.

\subsection{Covariance Analysis}

When analyzing MD trajectories, we are often interested in observing coordinated, or correlated motions. This is because global motions are likely to be involved in some functional mechanism. However, molecular trajectories typically feature complex motions along many axes and time scales which can often make it difficult to detect coordinated motions. Local rearrangements, vibrations, rotations, and random diffusion are examples of non-coordinated motions that likely do not contribute to a functional mechanism. The goal in MD trajectory covariance analysis is to obtain the axes of motion where atoms in the peptide of interest show a high degree of correlation which could be indicative of a global coordinated motion. 

Covariance analysis, or principal component analysis is a mathematical tool which isolates principal axes, or components of correlated motion by computing the covariance between atoms at every time point in the simulation. We compute the covariance for pairs $r_i$, $r+j$ of $N$ atoms  in $3$ dimensions resulting in a covariance matrix of size $3N$. $M$ is again a matrix of atomic weights.


\begin{equation}
C_{ij} = \bigg\langle M_{ii}^{\frac{1}{2}} (\vek{r_i}(t) - \langle \vek{r_i}(t) \rangle) M_{jj}^{\frac{1}{2}} (\vek{r_j}(t) - \langle \vek{r_j}(t) \rangle) \bigg\rangle
\end{equation}

The eigenvectors of the covariance matrix, $C$ define the set of orthogonal axes along which maximize variance. Note that $\langle \quad \rangle$ denotes a time average. Due to the constraints imposed by the backbone, only a limited number of eigenvectors are expected to contribute most to global movements.

\begin{equation}
R^{T}CR = diag(\lambda_1, \lambda_2, ... \lambda_{3N}) \text{\qquad where \quad} \lambda_1 \geq \lambda_2 \geq \lambda_{3N}
\end{equation}

Where R is the transformation matrix whose columns contain an eigenvector. Using this matrix to diagonalize C, we get a diagonalized C containing the set of eigenvalues $\lambda_i$ for every eigenvector in R along its main diagonal. The magnitude of the eigenvalue tells us the amount of variance captured by its corresponding eigenvector and can thus be used to guide our projection toward the major axes of motion.

If we wish to visualize motions along a particular axis and filter out motions along other axes, we can project the coordinates of each atom along a specific eigenvector. We can use following transformation to obtain the new set of coordinates $\vek{p}(t)$.

\begin{equation}
\vek{p}(t) = R^{T}M^{\frac{1}{2}}\vek{r}(t)
\end{equation}

The resulting trajectory lets us visualize motions along any component and is a useful tool for detecting coordinated structural changes. 

%Another useful application of the covariance matrix is for comparing the structural sampling of major modes in two different trajectories. Such comparisons can be used to assess the convergence of sampling, as well as for detecting major changes in conformational sampling. We define a quantity known as the subspace overlap which corresponds to the degree of similarity between two trajectories along their major components. This problem is equivalent to comparing two matrices of eigenvectors and the space that they span and so the subspace overlap is defined as follows:
%
%\begin{equation}
%\text{overlap}(\vek{v}, \vek{w}) \equiv \frac{1}{n} \sum_{i=1, j=1}^{n,n} (\vek{v} \times \vek{w}) ^ 2
%\end{equation}
%
%\begin{equation}
%d = \sqrt{tr((\sqrt{C_1} - \sqrt{C_2}))^2}
%\end{equation}
%
%\begin{equation}
%\text{normalized overlap} (C_1, C_2) = 1 - \frac{d}{\sqrt{tr(C_1) + tr(C_2)}}
%\end{equation}




%!TEX root = thesis_cgo.tex
\chapter{Conformational analysis of the $\gamma$-Tubulin carboxyl terminus}

In this chapter, we present a study on the conformational sampling of the $\gamma$-tubulin carboxyl terminus ($\gamma$-CT) as obtained through molecular dynamics simulations (MDS). hihihihi

\chapter{Conclusions}
\input{conclusion.tex}

\input{appendix.tex}
\bibHeading{References}
\bibliography{thesis_cgo}
\bibliographystyle{plain}




\end{document}


 






